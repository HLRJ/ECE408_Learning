%! Author = zhenxiang
%! Date = 23-3-25
\PassOptionsToPackage{quiet}{xeCJK}  % 抑制无意义的警告
% Preamble
\documentclass[11pt]{ctexart}

% Packages
\usepackage{amsmath}
% graphicx
\usepackage{graphicx}
% 页面设置
\usepackage{geometry}
\geometry{left=2.5cm, right=2.5cm, top=2.5cm, bottom=2.5cm}
% codes
\usepackage{ listings}
\usepackage{xcolor}
\usepackage{color}
\definecolor{mygreen}{rgb}{0,0.6,0}
\definecolor{mygray}{rgb}{0.5,0.5,0.5}
\definecolor{mymauve}{rgb}{0.58,0,0.82}
\lstset{ %
	backgroundcolor=\color{white},   % choose the background color; you must add \usepackage{color} or \usepackage{xcolor}
	basicstyle=\ttfamily,            % the size of the fonts that are used for the code
	breakatwhitespace=false,         % sets if automatic breaks should only happen at whitespace
	breaklines=true,                 % sets automatic line breaking
	captionpos=b,                    % sets the caption-position to bottom
	commentstyle=\ttfamily\color{mygreen},    
	% comment style
	deletekeywords={},               % if you want to delete keywords from the given language
	escapeinside={},                 % if you want to add LaTeX within your code
	extendedchars=true,              % lets you use non-ASCII characters; for 8-bits encodings only, does not work with UTF-8
	frame=single,                    % adds a frame around the code
	keepspaces=true,                 % keeps spaces in text, useful for keeping indentation of code (possibly needs columns=flexible)
	keywordstyle=\color{blue},       % keyword style
	language=C++,                    % the language of the code
	morekeywords={},                 % if you want to add more keywords to the set
	numbers=left,                    % where to put the line-numbers; possible values are (none, left, right)
	numbersep=5pt,                   % how far the line-numbers are from the code
	numberstyle=\tiny\color{mygray}, % the style that is used for the line-numbers
	rulecolor=\color{black},         % if not set, the frame-color may be changed on line-breaks within not-black text (e.g. comments (green here))
	showspaces=false,                % show spaces everywhere adding particular underscores; it overrides 'showstringspaces'
	showstringspaces=false,          % underline spaces within strings only
	showtabs=false,                  % show tabs within strings adding particular underscores
	stepnumber=1,                    % the step between two line-numbers. If it's 1, each line will be numbered
	stringstyle=\color{mymauve},     % string literal style
	tabsize=2,                       % sets default tabsize to 2 spaces
	title=\lstname                   % show the filename of files included with \lstinputlisting; also try caption instead of title
}

% Document
\begin{document}



\section{Prerequisites}
\begin{itemize}
	\item You have completed all week 1 lectures or videos
	\item You have completed Lab0 (MP0)
	\item \textbf{Chapter 2 }of the text book would also be helpful
\end{itemize}

\section{环境配置}
根据目录中的rai\_build.yml反推出makefile文件内容:

\begin{lstlisting}
	WB = ${WB_DIR}
	.DEFAULT_GOAL := cleanall
	.PHONY: clean cleanall
	all: template
	
	template: template.cu
	nvcc -std=c++11 -rdc=true -I $(WB) -c template.cu -o template.o
	nvcc -std=c++11 -o template template.o $(WB)/lib/libwb.so
	bash run_datasets
	clean:
	-rm -f template.o
	
	
	cleanall: clean
	rm -f template
	
	
\end{lstlisting}

\section{Instruction}
\begin{itemize}
	\item You should edit the code in `template.cu` to perform the following:
	\begin{itemize}
		\item[$\circ$] Allocate device memory
		\item[$\circ$] Copy host memory to device
		\item[$\circ$] Initialize thread block and kernel grid dimensions
		\item[$\circ$] Invoke CUDA kernel
		\item[$\circ$] Copy results from device to host
		\item[$\circ$] Free device memory
		\item[$\circ$]  Write the CUDA kernel
	\end{itemize}
\end{itemize}

Instructions about where to place each part of the code is
demarcated by the `//@@` comment lines.


\end{document}